\documentclass[12pt]{article}
\usepackage[T1]{fontenc}
\usepackage[utf8]{inputenc}
\usepackage[brazil]{babel}
\usepackage{hyperref}
\usepackage[a4paper,top=3.5cm,left=3cm,right=3cm,bottom=2.5cm]{geometry}

%configurando identação e separação de parágrafos
\parindent 1.27cm
\parskip   6pt

%configurando os hyperlinks
\hypersetup{
    colorlinks=true,
    linkcolor=green,
    filecolor=magenta,      
    urlcolor=blue,
}

%títulos,autor e data
\title{\textbf{Um pouco sobre Ransomwares}}
\author{Gustavo Lopes Rodrigues}
\date{Novembro de 2020}

\begin{document}
    
    \maketitle

    \textbf{\emph{Ransomware}} é um tipo de ataque criptográfico
    que: ao ganhar acesso do computador da vítima, o software 
    criptografa os dados da máquina. Para restaurar os dados comprometidos,
    a usuário receberá instruções em como pagar um valor X
    (geralmente em \textbf{criptomoedas}, pois elas deixam menos 
    vestígios, logo, são mais difíceis de serem re-encontradas) 
    para receber a chave de decriptografia e ganhar acesso novamente. 
    Infelizmente, existe casos onde os sofredores não recebem seus dados, 
    ou pior ainda, aqueles que fizeram o pagamento tem chance de
    serem alvos novamente de futuros ataques.

    Existe diferentes formas na qual um Ransomware pode ganhar 
    acesso a um computador:

    \begin{itemize}
        \item \textbf{\emph{Pishing}} - Ciberataques que envolve
        o uso de e-mails, mensagem de celulares(SMS), onde o recipiente
        é enganado a baixar um arquivo ou acessar um link. O nome "Pishing" é 
        derivado da palavra "Fishing" que é a palavra em inglês para "pescar".
        Ao ser "pescado", o hacker acaba de ganhar um ponto de acesso ao seu 
        computador, assim instalado o Ransomware.
    
        \item \textbf{\emph{Exploit Kit}} - ao contrário do Pishing, em vez
        de "pescar" a vítima, o Exploit Kit aproveita-se de falhas de programas
        ou até de sistemas, para acessar os computadores de pessoas distraídas.
        Um exemplo disso está no famoso ataque \href{https://olhardigital.com.br/especial/wannacry/}{\textbf{\emph{WannaCry}}} 
        que infectou computadores ao redor do mundo, por causa de uma falha nos 
        sistemas operacionais Windows.
    \end{itemize}

    %Iniciando referências
    \begin{thebibliography}{3}
        
        \bibitem{Ransomware} 
        \url{https://www.csoonline.com/article/3236183/what-is-ransomware-how-it-works-and-how-to-remove-it.html}
        
        \bibitem{pishing} 
        \url{https://br.malwarebytes.com/phishing/}

        \bibitem{Ransomware2} 
        \url{https://www.trendmicro.com/vinfo/us/security/definition/ransomware}
        
    \end{thebibliography}

\end{document}
